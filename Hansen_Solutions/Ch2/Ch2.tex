\documentclass[14pt]{extreport}
\usepackage[utf8]{inputenc}
\usepackage[a4paper, total={7in, 10in}]{geometry}
\usepackage{graphicx}
\usepackage{amsmath}
\usepackage{amssymb}
\usepackage{enumitem}

\newcommand{\ddfrac}[2]{\frac{\displaystyle #1}{\displaystyle #2}}
\newcommand{\eq}[0]{\llap{\(\Leftrightarrow\)\qquad}}
\newcommand{\answer}[0]{\medskip \textbf{Answer:} \medskip \\}
\newcommand{\union}[0]{\cup}
\newcommand{\intersect}[0]{\cap}
\newcommand{\sumn}[0]{\sum\limits_{i=1}^n}
\newcommand{\limn}[0]{\lim_{n \to \infty}}
\newcommand{\limt}[0]{\lim_{t \to \infty}}

\title{Solutions to Chapter 2 Exercises}
\author{Cangyuan Li}
\date{\today}

\begin{document}

\maketitle

\begin{enumerate}
    \item [\textbf{2.17}] 
    Let \(X\) be a random variable with \(\mu = \mathbb{E}[X]\) and \(\sigma^2 = var(X)\). 
    Define

    \begin{align*}
        g(x, \mu, \sigma^2) = \begin{pmatrix}
            x - \mu \\
            (x - \mu^2) - \sigma^2
        \end{pmatrix}
    \end{align*}

    Show that \(\mathbb{E}[g(X, \mu, \sigma)] = 0\) if and only if \(m = \mu\) and \(s = \sigma^2\).

    \answer
    Step 1: Show both set of conditions. Note that \(\mu = \mathbb{E}[X]\) and \(\sigma^2 = var(X)\).
    It is helpful to write in these terms in order to apply law of iterated expectations.

    \begin{align*}
        \mathbb{E}[g(X, m, s)] &= \mathbb{E}
            \begin{pmatrix}
                X - \mathbb{E}[X] \\
                (X - \mathbb{E}[X])^2 - var(X)
            \end{pmatrix} \\
            &= \begin{pmatrix}
                \mathbb{E}[X] - \mathbb{E}[\mathbb{E}[X]] \\
                (\mathbb{E}[X - \mathbb{E[X]}])^2 - \mathbb{E}[var(X)]
            \end{pmatrix} \\
            &= \begin{pmatrix}
                \mathbb{E}[X] - \mathbb{E}[X] \\
                \mathbb{E}[var(X)] - \mathbb{E}[var(X)]
            \end{pmatrix} \\
            &= 0
    \end{align*}

    Step 2:
    \medskip
    For example, if \(m \neq \mu\), then the law of iterated expectations would be violated.
\end{enumerate}

\end{document}