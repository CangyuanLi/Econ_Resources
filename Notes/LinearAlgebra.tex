\documentclass[14pt]{extreport}
\usepackage[utf8]{inputenc}
\usepackage[a4paper, total={7in, 10in}]{geometry}
\usepackage{graphicx}
\usepackage{amsmath}
\usepackage{amssymb}
\usepackage{enumitem}

\newcommand{\ddfrac}[2]{\frac{\displaystyle #1}{\displaystyle #2}}
\newcommand{\eq}[0]{\llap{\(\Leftrightarrow\)\qquad}}
\newcommand{\answer}[0]{\medskip \textbf{Answer:} \medskip \\}
\newcommand{\union}[0]{\cup}
\newcommand{\intersect}[0]{\cap}
\newcommand{\sumn}[0]{\sum\limits_{i=1}^n}
\newcommand{\limn}[0]{\lim_{n \to \infty}}
\newcommand{\limt}[0]{\lim_{t \to \infty}}

\title{Linear Algebra Review}
\author{Cangyuan Li}
\date{\today}


\begin{document}
    
\maketitle

\section{Basics}

\subsection{Dot Product Rules}

Where \( v \) and \( u \) are vectors, the dot product can be expressed as 

\begin{align*}
    v \cdot u = v'u = \lVert v \rVert \lVert u \rVert \cos \theta = \sumn v_i u_i,
\end{align*}

where \( \lVert v \rVert \) is the geometric length of the vector, \( \sqrt{v \cdot v} \).

\begin{enumerate}
    \item \( v \cdot u = u \cdot v \)
    \item \( v \cdot (u + v) = v \cdot u + v \cdot v \)
    \item \( 2v \cdot u = v \cdot 2u \)
    \item \( (v + u) \cdot (v + u) = v \cdot v + u \cdot v + v \cdot u + u \cdot u \)
\end{enumerate}

\subsection{Matrix Multiplication}

Two matrices can only be multiplied if they share a dimension. So \( n \times k \times k \times n \)is valid, but \( n \times k \times n \times k \) is not valid. Multiplying matrices can be done column, row, or element-wise. For example, column-wise between a \( 3 \times 3 \) matrix and a \( 3 \times 1 \) vector looks like:

\begin{align*}
    \begin{bmatrix}
        a & b & c \\
        d & e & f \\
        g & h & i \\
    \end{bmatrix}
    \begin{bmatrix}
        x \\
        y \\
        z \\
    \end{bmatrix} &=
    x \begin{bmatrix}
        a \\
        b \\
        g \\
    \end{bmatrix} + 
    y \begin{bmatrix}
        b \\
        e \\
        h \\
    \end{bmatrix} + 
    z \begin{bmatrix}
        c \\
        f \\
        i \\
    \end{bmatrix}
\end{align*}

\begin{itemize}
    \item By entry: \( AB_{ij} = \text{row} A_i \cdot \text{column} B_j \)
    \item By rows of \(A\): \( i^{th} \text{row of } AB = (i^{th} \text{row of } A)B \)
    \item By columns of \(B\): \( j^{th} \text{column of } AB = A(j^{th} \text{column of} B) \)
\end{itemize}

\section{Row-Reduced Echelon Form}




\end{document}